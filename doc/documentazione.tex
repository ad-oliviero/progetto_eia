\documentclass{article}
\usepackage{graphicx}
\usepackage{listings}
\usepackage{xcolor}
\usepackage{hyperref}
\usepackage{geometry}
\geometry{a4paper, margin=1in}

\title{Project Documentation for Elementi di Intelligenza Artificiale}
\author{Group Members: Name1, Name2, Name3}
\date{\today}

\begin{document}

\maketitle

\tableofcontents

\section{Obiettivo del progetto}
Il progetto si propone di applicare algoritmi di ricerca a dataset reali per valutare l'efficacia e l'efficienza di tali algoritmi. In particolare, utilizzeremo il dataset disponibile su \url{https://snap.stanford.edu/data/} per sperimentare e confrontare diversi algoritmi di ricerca.

\section{Descrizione delle metodologie e tecniche adoperate}
Per la realizzazione del progetto, abbiamo utilizzato il linguaggio di programmazione Rust. Il progetto è organizzato come segue:
\begin{itemize}
  \item \texttt{Cargo.toml} - File di configurazione per il gestore di pacchetti di Rust, Cargo.
  \item \texttt{download-datasets.sh} - Script shell per scaricare i dataset.
  \item \texttt{LICENSE} - Informazioni sulla licenza del progetto.
  \item \texttt{README.md} - File Readme con una panoramica del progetto.
  \item \texttt{run.py} - Script Python per eseguire il progetto.
  \item \texttt{src/} - Directory contenente il codice sorgente in Rust.
        \begin{itemize}
          \item \texttt{main.rs} - Punto di ingresso principale per l'applicazione Rust.
          \item \texttt{macros.rs} - Contiene definizioni di macro utilizzate nel progetto.
          \item \texttt{args.rs} - Gestisce gli argomenti della riga di comando.
          \item \texttt{problem/} - Directory contenente i moduli relativi ai problemi di ricerca.
                \begin{itemize}
                  \item \texttt{mod.rs} - File di modulo per la directory problem.
                  \item \texttt{graph.rs} - Modulo per le strutture e funzioni relative ai grafi.
                  \item \texttt{node.rs} - Modulo per le strutture e funzioni relative ai nodi.
                \end{itemize}
        \end{itemize}
\end{itemize}

\subsection{Tecnologie Utilizzate}
Abbiamo utilizzato diverse tecnologie per sviluppare e analizzare il progetto:
\begin{itemize}
  \item \textbf{Rust}: Il linguaggio di programmazione principale utilizzato per implementare gli algoritmi di ricerca.
  \item \textbf{Python}: Utilizzato per gli script di gestione e automazione, come il download dei dataset.
  \item \textbf{Massif}: Uno strumento di profiling di memoria che fa parte di Valgrind, utilizzato per analizzare l'uso della memoria del programma.
\end{itemize}

\section{Dataset}
I dataset utilizzati sono stati scaricati dal sito dell'università di Stanford (\url{https://snap.stanford.edu/data/}).
Per scaricare i dataset, è possibile procedere manualmente recandosi alle reciproche pagine sul sito, oppure utilizzare lo script shell fornito nel progetto:
\begin{verbatim}
$ chmod +x ./download-datasets.sh
$ ./download-datasets.sh
\end{verbatim}
Lo script utilizza \texttt{wget} ed è scritto per sistemi UNIX \& UNIX-like.

\subsection{Dataset Utilizzati}
\begin{table}[h]
  \centering
  \begin{tabular}{|l|r|r|l|r|}
    \hline
    Nome                                                                                      & Nodi    & Archi    & Tipo       & Dimensione \\
    \hline
    \href{https://snap.stanford.edu/data/soc-sign-bitcoin-alpha.html}{soc-sign-bitcoin-alpha} & 3783    & 24186    & Labled     & 152KB      \\
    \href{https://snap.stanford.edu/data/email-Enron.html}{email-Enron}                       & 36692   & 183831   & Undirected & 1.1MB      \\
    \href{https://snap.stanford.edu/data/com-Youtube.html}{com-Youtube}                       & 1134890 & 2987624  & Undirected & 11MB       \\
    \href{https://snap.stanford.edu/data/roadNet-CA.html}{roadNet-CA}                         & 1965206 & 2766607  & Directed   & 18MB       \\
    \href{https://snap.stanford.edu/data/as-Skitter.html}{as-Skitter}                         & 1696415 & 11095298 & Undirected & 33MB       \\
    \href{https://snap.stanford.edu/data/cit-Patents.html}{cit-Patents}                       & 3774768 & 16518948 & Directed   & 85MB       \\
    \href{https://snap.stanford.edu/data/com-LiveJournal.html}{com-LiveJournal}               & 3997962 & 34681189 & Undirected & 124MB      \\
    \hline
  \end{tabular}
  \caption{Dataset Utilizzati}
\end{table}

I dataset contengono alcune informazioni nelle prime righe. Sono in formato \texttt{txt} e le proprie righe sono formate da due numeri (Nodo Sinistro e Nodo Destro), ad eccezione del dataset \href{https://snap.stanford.edu/data/soc-sign-bitcoin-alpha.html}{soc-sign-bitcoin-alpha}, che è in formato \texttt{csv} con le colonne: \textbf{SOURCE} (id del nodo Sinistro), \textbf{TARGET} (id del nodo Destro), \textbf{RATING} (il costo delle azioni), e \textbf{TIME} (non rilevante).

\section{Risultati sperimentali}
Abbiamo applicato gli algoritmi di ricerca ai dataset forniti al link \url{https://snap.stanford.edu/data/}. I risultati ottenuti sono stati valutati in termini di efficacia ed efficienza, come descritto di seguito.

L'esecuzione in totale è durata 14.821 secondi

\subsection{email-Enron.txt.gz}
Tipo di Grafo: Directed

Durata caricamento: 1.283s

Nodi cercati: 3394 e 404

\begin{table}[h]
\centering
\begin{tabular}{|l|l|r|r|r|}
\hline
\textbf{Algoritmo} & \textbf{Risultato} & \textbf{Profondità} & \textbf{Costo} & \textbf{Tempo} \\
 \hline
bi-directional & Trovato & 3 & 0 & 12.238095s \\
\hline
\end{tabular}
\caption{email-Enron.txt.gz}
\end{table}


\section{Codice Sviluppato}
Il codice sviluppato è stato caricato insieme alla documentazione del progetto e può essere consultato nei file allegati.

\end{document}

